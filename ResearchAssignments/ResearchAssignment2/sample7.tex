%% Beginning of file 'sample7.tex'
%%
%% Version 7. Created January 2025.  
%%
%% AASTeX v7 calls the following external packages:
%% times, hyperref, ifthen, hyphens, longtable, xcolor, 
%% bookmarks, array, rotating, ulem, and lineno 
%%
%% RevTeX is no longer used in AASTeX v7.
%%
\documentclass[linenumbers,trackchanges]{aastex7}
%% This initial command takes arguments that can be used to easily modify 
%% the output of the compiled manuscript. Any combination of arguments can be 
%% invoked like this:
%%
%% \documentclass[argument1,argument2,argument3,...]{aastex7}
%%
%% Six of the arguments are typestting options. They are:
%%
%%  twocolumn   : two text columns, 10 point font, single spaced article.
%%                This is the most compact and represent the final published
%%                derived PDF copy of the accepted manuscript from the publisher
%%  default     : one text column, 10 point font, single spaced (default).
%%  manuscript  : one text column, 12 point font, double spaced article.
%%  preprint    : one text column, 12 point font, single spaced article.  
%%  preprint2   : two text columns, 12 point font, single spaced article.
%%  modern      : a stylish, single text column, 12 point font, article with
%% 		  wider left and right margins. This uses the Daniel
%% 		  Foreman-Mackey and David Hogg design.
%%
%% Note that you can submit to the AAS Journals in any of these 6 styles.
%%
%% There are other optional arguments one can invoke to allow other stylistic
%% actions. The available options are:
%%
%%   astrosymb    : Loads Astrosymb font and define \astrocommands. 
%%   tighten      : Makes baselineskip slightly smaller, only works with 
%%                  the twocolumn substyle.
%%   times        : uses times font instead of the default.
%%   linenumbers  : turn on linenumbering. Note this is mandatory for AAS
%%                  Journal submissions and revisions.
%%   trackchanges : Shows added text in bold.
%%   longauthor   : Do not use the more compressed footnote style (default) for 
%%                  the author/collaboration/affiliations. Instead print all
%%                  affiliation information after each name. Creates a much 
%%                  longer author list but may be desirable for short 
%%                  author papers.
%% twocolappendix : make 2 column appendix.
%%   anonymous    : Do not show the authors, affiliations, acknowledgments,
%%                  and author contributions for dual anonymous review.
%%  resetfootnote : Reset footnotes to 1 in the body of the manuscript.
%%                  Useful when there are a lot of authors and affiliations
%%		    in the front matter.
%%   longbib      : Print article titles in the references. This option
%% 		    is mandatory for PSJ manuscripts.
%%
%% Since v6, AASTeX has included \hyperref support. While we have built in 
%% specific %% defaults into the classfile you can manually override them 
%% with the \hypersetup command. For example,
%%
%% \hypersetup{linkcolor=red,citecolor=green,filecolor=cyan,urlcolor=magenta}
%%
%% will change the color of the internal links to red, the links to the
%% bibliography to green, the file links to cyan, and the external links to
%% magenta. Additional information on \hyperref options can be found here:
%% https://www.tug.org/applications/hyperref/manual.html#x1-40003
%%
%% The "bookmarks" has been changed to "true" in hyperref
%% to improve the accessibility of the compiled pdf file.
%%
%% If you want to create your own macros, you can do so
%% using \newcommand. Your macros should appear before
%% the \begin{document} command.
%%
\newcommand{\vdag}{(v)^\dagger}
\newcommand\aastex{AAS\TeX}
\newcommand\latex{La\TeX}
%%%%%%%%%%%%%%%%%%%%%%%%%%%%%%%%%%%%%%%%%%%%%%%%%%%%%%%%%%%%%%%%%%%%%%%%%%%%%%%%
%%
%% The following section outlines numerous optional output that
%% can be displayed in the front matter or as running meta-data.
%%
%% Running header information. A short title on odd pages and 
%% short author list on even pages. Note that this
%% information may be modified in production.
%%\shorttitle{AASTeX v7 Sample article}
%%\shortauthors{The Terra Mater collaboration}
%%
%% Include dates for submitted, revised, and accepted.
%%\received{February 1, 2025}
%%\revised{March 1, 2025}
%%\accepted{\today}
%%
%% Indicate AAS Journal the manuscript was submitted to.
%%\submitjournal{PSJ}
%% Note that this command adds "Submitted to " the argument.
%%
%% You can add a light gray and diagonal water-mark to the first page 
%% with this command:
%% \watermark{text}
%% where "text", e.g. DRAFT, is the text to appear.  If the text is 
%% long you can control the water-mark size with:
%% \setwatermarkfontsize{dimension}
%% where dimension is any recognized LaTeX dimension, e.g. pt, in, etc.
%%%%%%%%%%%%%%%%%%%%%%%%%%%%%%%%%%%%%%%%%%%%%%%%%%%%%%%%%%%%%%%%%%%%%%%%%%%%%%%%
%%
%% Use this command to indicate a subdirectory where figures are located.
%%\graphicspath{{./}{figures/}}
%% This is the end of the preamble.  Indicate the beginning of the
%% manuscript itself with \begin{document}.

\begin{document}
\title{Tidal Transformation of Satellites (M33: Stellar Structure)}
%% A significant change from AASTeX v6+ is in the author blocks. Now an email
%% address is required for each author. This means that each author requires
%% at least one of the following:
%%
%% \author
%% \affiliation
%% \email
%%
%% If these three commands are not available for each author, the latex
%% compiler will issue an error and if you force the latex compiler to continue,
%% it will generate an incomplete pdf.
%%
%% Multiple \affiliation commands are allowed and authors can also include
%% an optional \altaffiliation to indicate a status, i.e. Hubble Fellow. 
%% while affiliations are indexed as footnotes, altaffiliations are noted with
%% with a non-numeric footnote that is set away from the numeric \affiliation 
%% footnotes. NOTE that if an \altaffiliation command is used it must 
%% come BEFORE the \affiliation call, right after the \author command, in 
%% order to place the footnotes in the proper location. Because non-numeric
%% symbols are used, \altaffiliation should be used sparingly.
%%
%% In v7 the \author command takes an optional argument which provides 
%% additional metadata about the author. Authors can provide the 16 digit 
%% ORCID, the surname (family or last) name, the given (first or fore-) name, 
%% and a name suffix, e.g. "Jr.". The syntax is:
%%
%% \author[orcid=0000-0002-9072-1121,gname=Gregory,sname=Schwarz]{Greg Schwarz}
%%
%% This name metadata in not shown, it is only for parsing by the peer review
%% system so authors can be more easily identified. This name information will
%% also be sent to the publisher so they can include it in the CROSSREF 
%% metadata. Including an orcid will hyperlink the author name to the 
%% author's ORCID page. Note that  during compilation, LaTeX will do some 
%% limited checking of the format of the ID to make sure it is valid. If 
%% the "orcid-ID.png" image file is  present or in the LaTeX pathway, the 
%% ORCID icon will appear next to the authors name.
%%
%% Even though emails are now required for each author, the \email does not
%% produce output in the compiled manuscript unless the optional "show" command
%% is used. For example,
%%
%% \email[show]{greg.schwarz@aas.org}
%%
%% All "shown" emails are show in the bottom left of the first page. Due to
%% space constraints, only a few emails should be shown. 
%%
%% To identify a corresponding author, use the \correspondingauthor command.
%% The command appends "Corresponding Author: " to the argument it appears at
%% the bottom left of the first page like the output from \email. 
\author{Colin Miller}\affiliation{University of Arizona, Tucson, AZ}\email{colinmiller1@arizona.edu}

\section{Introduction}
The proposed topic is the tidal transformation of satellites in relation to the stellar structure of a galaxy. The evolution happens due to the tidal forces of a massive host galaxy close to many satellites (e.g. dwarf galaxies). The massive host then induces variation in the stellar structure of the galaxy, including the central bar and halo of the satellites. Many times in galaxy groups there is an offset of the bar in the main host galaxy compared to its satellites.  

It matters because the same principle applies to bigger galaxy pass-by's. In so much as, how galaxies evolve as they merge or pass by each other, and how this affects the evolution both the galaxies and smaller satellite galaxies. The physics and simulations can then predict how galaxies evolved in the past to the present day and the future of the said galaxies' internal structure. Resonant effects on the orbital motion of satellites and the orientation of their angular momentum are strong indicators of how satellites will evolve as it merges with other satellites outside the host galaxy. It can also help to describe the offset of stellar and gaseous disks in galaxies. Lastly, it can help predict where star formation will occur in galaxies and potentially at what rate. Disc galaxies commonly show asymmetric features in their morphology, such as warps and lopsidedness. 

Our current understanding of galaxy evolution as it pertains to internal stellar structure is as follows. The bar center of galaxies is coincident with the dynamical center of the said galaxies. This can mean that the bar is not displaced and the stellar disks are instead from the dynamical center. The gas disk, too, moves in conjunction with the stellar disk because of gravitational tidal effects during merges and pass-byes. In addition, tidal interaction effects are strongest for the exact pro-grade orientation in the disk of the dwarf galaxies. In turn, there is a decrease in the satellite's orbital velocity. In the case of exact retrograde, it appears there is no strong evolution (e.g. satellite’s stellar component does not form a bar and remains a disk). This is due to the switch in the sign of the Coriolis force acting on the satellite from the host galaxy. Tidal stirring is another phenomenon, where during a merger or pass-bye, mass is lost on the satellite, orbital velocity decreases, and stirring occurs when gas is compressed in the galaxy, which in turn triggers bursts of star formation in the satellite. As satellites approach peri-center, low surface brightness satellites lose most of their dark and stellar mass. This is due to their low-density halos and large disks. Thus the bar becoming weakly unstable. High surface brightness satellites undergo some tidal stripping, and their more self-gravitating disks develop strong central bars. There is strong correlation between internal galaxy properties, such as central stellar surface density and disc radial extension with the strength of lopsided modes. The majority of lopsided galaxies have lower central surface densities and more extended discs than symmetric galaxies. Lopsided galaxies tend to live in asymmetric dark matter halos with high spin, indicating strong galaxy–halo connections in late-type lopsided galaxies. For disc galaxies it can induce the redistribution of stellar mass due to angular momentum transport and the modulation of hosts star formation histories. In addition, the internal torques induced by such m = 1 modes can result in the loss of angular momentum by the host gaseous disc, thus affecting the growth of the central supermassive black hole in the center.

There are many open questions to be studied within the evolution of the internal stellar structure. Such examples show how bulge-less galaxies, with nuclear clusters that are off from dynamical center, formed in the past and how they will continue in the future. It is unknown how the population of blue compact dwarfs identified by Guzman et al. (1997) at intermediate redshifts formed and it's speculated that tidal stirring may have needed to occur. It is yet still unknown how up to 30 per cent of late-type galaxies display a global non-axisymmetric lopsided mass distribution. 

\section{The Proposal} 

\subsection{Proposal}

\subsection{Methods}

\subsection{Hypothesis}

\end{document}